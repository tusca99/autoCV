%-----------------------------------------------------------------------------------------------------------------------------------------------%
%    CV template adapted for Alessio Tusca
%-----------------------------------------------------------------------------------------------------------------------------------------------%
\documentclass[a4paper,10pt]{article}

% Packages
\usepackage[utf8]{inputenc}
\usepackage[T1]{fontenc}
\usepackage{url}
\usepackage{parskip}
\RequirePackage{color}
\RequirePackage{graphicx}
\usepackage[dvipsnames]{xcolor}
\usepackage[scale=0.9]{geometry}
\usepackage{tabularx}
\usepackage{enumitem}
\usepackage{titlesec}
\usepackage{multicol}
\usepackage{multirow}
\usepackage[style=authoryear,sorting=ynt, maxbibnames=2]{biblatex}
\usepackage[unicode, draft=false]{hyperref}
\usepackage{fontawesome5}
\usepackage{colortbl}
\addbibresource{citations.bib}

% Layout tweaks
\definecolor{linkcolour}{rgb}{0,0.2,0.6}
\hypersetup{colorlinks,breaklinks,urlcolor=linkcolour,linkcolor=linkcolour}
\setlength\bibitemsep{1em}
\newcolumntype{C}{>{\centering\arraybackslash}X}
\newlength{\fullcollw}
\setlength{\fullcollw}{0.47\textwidth}
\titleformat{\section}{\Large\scshape\raggedright}{}{0em}{}[\titlerule]
\titlespacing{\section}{0pt}{8pt}{8pt}

% Job environments
\newenvironment{jobshort}[2]
    {
    \begin{tabularx}{\linewidth}{@{}l X r@{}}
        		\textbf{#1} & \hfill &  #2 \\[3.75pt]
    \end{tabularx}
    }
    {
    }

\newenvironment{joblong}[2]
    {
    \begin{tabularx}{\linewidth}{@{}l X r@{}}
        		\textbf{#1} & \hfill &  #2 \\[3.75pt]
    \end{tabularx}
    \begin{minipage}[t]{\linewidth}
    \begin{itemize}[nosep,after=\strut, leftmargin=1em, itemsep=3pt,label=--]
    }
    {
    \end{itemize}
    \end{minipage}    
    }

% Document
\begin{document}
\pagestyle{empty}

% Header
\begin{tabularx}{\linewidth}{@{} C @{} }
\Huge{Alessio Tuscano} \\[5pt]
\href{https://github.com/tusca99}{\raisebox{-0.05\height}\faGithub\ tusca99} \ $|$ \ 
\href{https://www.linkedin.com/in/alessio-tuscano-188b5315a}{\raisebox{-0.05\height}\faLinkedin\ alessio-tuscano} \ $|$ \ 
\href{mailto:alessio.tuscano@studenti.unipd.it}{\raisebox{-0.05\height}\faEnvelope \ alessio.tuscano@studenti.unipd.it} \ $|$ \ 
\href{mailto:alessio@tuscano.it}{\raisebox{-0.05\height}\faEnvelope \ alessio@tuscano.it} \\
\end{tabularx}

\section{Summary}
{Physics master's student (M.Sc. Physics of Data, University of Padova) with a strong background in computational physics, high-performance computing, and scientific software development. Experienced in accelerating and optimizing existing codebases for large-scale simulations (CPU/GPU), with solid theoretical foundations in statistical mechanics, complex systems, and data analysis. Skilled in parallel programming, numerical methods, and modern approaches to inference and modeling.
}

% Work Experience
\section{Work Experience}

\begin{joblong}{Research guest collaborator --- INFN (FEROCE project)}{2025--present}
\item Currently working on GPU accelerated data pipelines and computational kernels in containers, enabling easy scalability and deployment for FEROCE project, related proceeding: \url{https://agenda.infn.it/event/44098/contributions/253504/}.
\end{joblong}

\begin{joblong}{Typesetter}{2025--present}
\item Typesetting and related automations for academic publications and collaborations at SISSA-medialab for scientific journals such as JCAP.
\end{joblong}

\section{Education \& Languages}
\begin{tabularx}{\linewidth}{@{}l X l X@{}}
2024--present & M.Sc. Physics of Data, \textbf{University of Padova} & \hfill \textbf{Italian} & Native \\
2018--2024 & B.Sc. Physics, \textbf{University of Padova} & \hfill \textbf{English} & Professional/Fluent \\
\end{tabularx}

% Projects
\section{Projects}

\begin{tabularx}{\linewidth}{ @{}l r@{} }
        \textbf{Bachelor thesis --- Water's surface potential reproduction with Neural Network} & \hfill \href{https://github.com/tusca99/Bachelor-Thesis}{GitHub} \\[3.75pt]
\multicolumn{2}{@{}X@{}}{\small Implemented numerical exact method (DFT) simulations in a distributed environment to create a dataset for a fine regression task with a Neural Network; code and thesis (in Italian only) available on GitHub.}  \\
\arrayrulecolor{gray!30}\hline        
        \textbf{N-body simulation} & \hfill \href{https://github.com/tusca99/N-body-simulation}{GitHub} \\[3.75pt]
\multicolumn{2}{@{}X@{}}{\small High-performance, modular C++/CUDA framework for astrophysical N-body simulations. Supports GPU acceleration, OpenMP parallelism, flexible initialization, and interactive visualization. Extensible codebase for research and experimentation; full documentation and code available on GitHub.}  \\
\arrayrulecolor{gray!30}\hline
        \textbf{Fine--Grained Vehicle Recognition with CNNs} & \hfill \href{https://github.com/tusca99/Cars-classification-using-CNN}{GitHub} \\[3.75pt]
\multicolumn{2}{@{}X@{}}{\small Designed and trained CNN and Siamese models on the CompCars dataset for vehicle classification and verification under class imbalance. Compared Cross-Entropy, Focal, and Contrastive losses.}  \\
\arrayrulecolor{gray!30}\hline
    \textbf{Hopfield Neural Network} & \hfill \href{https://github.com/tusca99/Hopfield-Model}{GitHub} \\[3.75pt]
\multicolumn{2}{@{}X@{}}{\small Implemented a Hopfield neural network for pattern recognition and image reconstruction on synthetic 2D patterns and MNIST, evaluating performance under noise and alternative update rules.}  \\
\arrayrulecolor{gray!30}\hline
    \textbf{Distributed K-Means with Dask} & \hfill \href{https://github.com/tusca99/Distributed-K-Means-Dask}{GitHub} \\[3.75pt]
\multicolumn{2}{@{}X@{}}{\small Implemented and benchmarked a distributed K-Means parallel algorithm with Dask, extending the dask-ml implementation with weighted centroid updates. Configured VM clusters via SSH, optimized memory/thread allocation, and analyzed scalability on large datasets (KDD Cup 1999).}  \\
\arrayrulecolor{gray!30}\hline
    \textbf{Simulation-Based Inference for a 2D random walk model} & \hfill \href{https://github.com/tusca99/PoD-LACP-B}{GitHub} \\[3.75pt]
\multicolumn{2}{@{}X@{}}{\small Applied simulation-based inference with neural posterior estimation to recover parameters of a stochastic 2D random walk, validated on an analytically solvable quartic model, and extended to a red blood cell model not accessible analytically. Optimized simulation performance with Numba to achieve C-like speed.}  \\
\arrayrulecolor{gray!30}\hline
    \textbf{Bayesian Analysis of ARPAV time series} & \hfill \href{https://github.com/tusca99/Bayesian-Analysis-ARPAV-Temperature-Time-Series}{GitHub} \\[3.75pt]
\multicolumn{2}{@{}X@{}}{\small Conducted a Bayesian analysis of ARPAV temperature and precipitation time series using MCMC methods to estimate trends and assess model fit, and applied SARIMA models for time series forecasting.}  \\
%\arrayrulecolor{gray!30}\hline
\end{tabularx}

% Skills
\section{Skills}
\begin{tabularx}{\linewidth}{@{}l X@{}}
Programming &  \normalsize{C/C++, Python, CUDA, OpenGL, Bash, R, basic HDL (Verilog/VHDL)}\\
HPC and parallelism  &  \normalsize{Asyncio, Dask, Numba, MPI, OpenMP, GPU acceleration, profiling}\\
Tools and infra &  \normalsize{Git, Docker, Kubernetes, Linux, FreeBSD}\\
\end{tabularx}

\vfill
\center{\footnotesize Last updated: \today}

\end{document}
