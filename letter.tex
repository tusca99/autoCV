\documentclass[a4paper,11pt]{article}
\usepackage[utf8]{inputenc}
\usepackage[T1]{fontenc}
\usepackage[scale=0.77]{geometry}
\usepackage{tabularx}
\usepackage{hyperref}
\usepackage{fontawesome5}
\usepackage{xcolor}

% Layout tweaks
\definecolor{linkcolour}{rgb}{0,0.2,0.6}
\hypersetup{colorlinks,breaklinks,urlcolor=linkcolour,linkcolor=linkcolour}


\begin{document}

% Header (come nel CV)
\begin{flushright}
\textbf{Alessio Tuscano}\\
\href{mailto:alessio.tuscano@studenti.unipd.it}{\raisebox{-0.05\height}\faEnvelope \ alessio.tuscano@studenti.unipd.it} \\
\end{flushright}

\vspace{.5em}
\noindent
\textbf{Date:} \today

\vspace{.5em}
\noindent
\textbf{Subject:} Application for Erasmus+ Traineeship

\vspace{1em}
\noindent
Dear Selection Committee,

My name is Alessio Tuscano and I am currently enrolled in the Master’s Degree in Physics of Data at the University of Padua. I am applying for the Erasmus+ Traineeship Programme because I wish to further develop my skills and contribute to research in an international and collaborative environment, ideally focused on computational physics, scientific computing, and complex systems.

During my studies I cultivated a strong interest in computational physics. Starting from my bachelor thesis, and following through numerous group and solo projects, I realized I want to be a strong support for researchers in many fields, while also being able to do some research myself in computational physics.

Some of the most formative experiences are my Bachelor Degree and in particular the Bachelor Thesis, there I learned for the first time how to build a scientific dataset ready for deep learning applications. Also more recently, along with the group projects and experiences in the Master Degree I did a bigger project on my own again, more as a benchmark of my current capabilities (at least in the recent year) and to learn new coding practices and manage bigger codebases for better maintainability and modularity, also while keeping state-of-the-art performances.

To put things very simply, I like making slow things (particularly code) fast and efficient, and helping researchers achieve their goals faster, or at all in some rare cases like the Simulation-Based-Inference project. It simply was not feasible to do that research due to the large simulation data needed for inference training, and accelerating the simulations by orders of magnitude made it possible in matter of minutes--hours of computation rather than days--weeks.

%% The collaboration with INFN on the FEROCE project is another instance of a new challenge for me where I want to give my best and learn to work at a bigger scale with more colleagues, a bigger project and also a bigger work even on my own part, and I truly wish to take as much as I can from this experience too. These experiences have strengthened, and strengthening my ability to work with scientific software in multiple research fields, accelerate existing codebases, and approach research questions with both rigor and creativity.

I believe that working in an international research group would allow me to broaden my perspective, learn new methodologies, and contribute actively to collaborative projects. It will be a key aspect in the start of my career as a researcher in HPC and computational physics.

Beyond the technical and academic aspects, I am also eager to immerse myself in a new cultural and scientific context. Having already experienced the value of collaboration and exchange during my studies in Padua, I am enthusiastic about the opportunity to take this further through an Erasmus+ traineeship, working alongside researchers from diverse backgrounds. I want to experience a different city than my hometown: since the start of my universities studies I commuted from a town one hour and a half far from the university. I grew to like this setting, but I also understood that to really live a city you have to give it more than what I gave to Padua. So I'd like to have this abroad experience to live fully both the lab/research activity and the city I will happen to be.

With this traineeship, I hope to contribute my computational and theoretical skills to a stimulating research project. I am flexible regarding the timing and duration of the traineeship as long as the research is stimulating enough and I do a statisfying job both for me and the team.

Thank you very much for your consideration.


\vspace{1em}
\noindent
Sincerely,\\[.4em]
Alessio Tuscano

\end{document}
