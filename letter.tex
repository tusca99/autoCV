\documentclass[a4paper,11pt]{article}
\usepackage[utf8]{inputenc}
\usepackage[T1]{fontenc}
\usepackage[scale=0.77]{geometry}
\usepackage{tabularx}
\usepackage{hyperref}
\usepackage{fontawesome5}
\usepackage{xcolor}

% Layout tweaks
\definecolor{linkcolour}{rgb}{0,0.2,0.6}
\hypersetup{colorlinks,breaklinks,urlcolor=linkcolour,linkcolor=linkcolour}


\begin{document}

% Header (come nel CV)
\begin{flushright}
\textbf{Alessio Tuscano}\\
\href{mailto:alessio.tuscano@studenti.unipd.it}{\raisebox{-0.05\height}\faEnvelope \ alessio.tuscano@studenti.unipd.it} \\
\end{flushright}

\vspace{.5em}
\noindent
\textbf{Date:} \today

\vspace{.5em}
\noindent
\textbf{Subject:} Application for Erasmus+ Traineeship

\vspace{1em}
\noindent
Dear Selection Committee,

My name is Alessio Tuscano and I am currently pursuing a Master’s Degree in Physics of Data at the University of Padua. I am applying to the Erasmus+ Traineeship Programme to further advance my expertise and contribute to research in an international setting, with a particular interest in computational physics, high-performance computing (HPC), and scientific computing within complex systems.

%During my studies I developed a strong interest in computational physics, working with GPU acceleration (CUDA), neural networks and deep learning, and exploring parallel programming and more recently FPGA architectures. Through my bachelor thesis and various group and solo projects, I learned how to support research teams by optimizing scientific code and implementing modern computational methods across different fields.

%Some of the most formative experiences are my Bachelor Degree and in particular the Bachelor Thesis, there I learned for the first time how to build a scientific dataset ready for deep learning applications. The Bachelor Degree taught me how to approach multiple theoretical fields, and I chose this Master Degree to complement my theoretical knowledge with more computational and practical skills.
%Also more recently, along with the group projects and experiences in the Master Degree I did a bigger project on my own again, more as a benchmark of my current capabilities (at least in the recent year) and to learn new coding practices and manage bigger codebases for better maintainability and modularity, also while keeping state-of-the-art performances.

%To put things very simply, I like making slow things (particularly code) fast and efficient, and helping researchers achieve their goals faster, or at all in some rare cases like the Simulation-Based-Inference project. It simply was not feasible to do that research due to the large simulation data needed for inference training, and accelerating the simulations by orders of magnitude made it possible in matter of minutes--hours of computation rather than days--weeks.
During my studies, I developed a strong interest in computational physics and scientific programming, working with GPU acceleration (CUDA), neural networks, deep learning, parallel computing, and FPGA architectures. My bachelor thesis introduced me to building scientific datasets for machine learning, while my master’s projects allowed me to further integrate theoretical knowledge with advanced computational skills.

Through both individual and group projects, I have focused on optimizing scientific code and improving the efficiency of research workflows. I particularly enjoy transforming slow or complex simulations into fast, reliable tools that enable new scientific results—such as making simulation-based inference feasible by accelerating data generation from weeks to hours.

This combination of practical experience and theoretical foundation has shaped my approach: I strive to support research teams by making their computational work more effective, while continuing to learn and innovate in the field.

%% The collaboration with INFN on the FEROCE project is another instance of a new challenge for me where I want to give my best and learn to work at a bigger scale with more colleagues, a bigger project and also a bigger work even on my own part, and I truly wish to take as much as I can from this experience too. These experiences have strengthened, and strengthening my ability to work with scientific software in multiple research fields, accelerate existing codebases, and approach research questions with both rigor and creativity.

I believe that working in an international research group would allow me to broaden my perspective, learn new methodologies, and contribute actively to collaborative projects. It will be a key aspect in the start of my career as a researcher in HPC and computational physics.

Beyond academics, I am eager to immerse myself in a new cultural and scientific environment. Having always commuted from my hometown to Padua, I realized that truly experiencing a city requires living it fully. With this traineeship, I hope not only to grow through research and collaboration with international peers, but also to embrace daily life in a new city—making the most of both the lab and the local culture.

With this traineeship, I hope to contribute my computational and theoretical skills to a stimulating research project. I am flexible regarding the timing and duration of the traineeship as long as the research is stimulating enough and I do a satisfying job both for me and the team.

Thank you very much for your consideration.


\vspace{1em}
\noindent
Sincerely,\\[.4em]
Alessio Tuscano

\end{document}
