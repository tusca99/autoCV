\documentclass[a4paper,11pt]{article}
\usepackage[utf8]{inputenc}
\usepackage[T1]{fontenc}
\usepackage{geometry}
\usepackage{tabularx}
\usepackage{hyperref}
\usepackage{fontawesome5}
\usepackage{xcolor}

% Layout tweaks
\definecolor{linkcolour}{rgb}{0,0.2,0.6}
\hypersetup{colorlinks,breaklinks,urlcolor=linkcolour,linkcolor=linkcolour}


\begin{document}

% Header (come nel CV)
\begin{flushright}
\textbf{Alessio Tuscano}\\
\href{mailto:alessio.tuscano@studenti.unipd.it}{\raisebox{-0.05\height}\faEnvelope \ alessio.tuscano@studenti.unipd.it} \\
\end{flushright}

\vspace{.5em}
\noindent
\textbf{Date:} \today

\vspace{.5em}
\noindent
\textbf{Subject:} Application for Erasmus+ Traineeship

\vspace{1em}
\noindent
%% Dear Selection Committee,

%% My name is Alessio Tuscano and I am currently enrolled in the Master’s Degree in Physics of Data at the University of Padua. I am applying for the Erasmus+ Traineeship Programme because I wish to further develop my skills and contribute to research in an international and collaborative environment, ideally focused on computational physics, scientific computing, and complex systems.

%% Throughout my academic journey, I have cultivated a strong interest in the intersection between theoretical physics and advanced computational methods. My studies have allowed me to explore topics such as statistical mechanics, complex networks, machine learning, and High-Performance Computing, both from a theoretical and practical perspective. I have worked on several projects—often in collaboration with my peers—where I translated physical models into efficient simulation code, optimized algorithms for parallel architectures, and applied modern statistical data analysis techniques to scientific problems.

%% Some of the most formative experiences are my Bachelor Degree and in particular the Bachelor Thesis, there I learned for the first time how to build a scientific dataset ready for deep learning applications. Also more recently, along with the group projects and experiences in the Master Degree I did a bigger project on my own again, more as a benchmark of my current capabilities (at least in the recent year) and to learn new coding practices and manage bigger codebases for better maintainability and modularity, also while keeping state-of-the-art performances.

%% To put things very simply, I like making slow things (particularly code) fast and efficient, and helping researchers achieve their goals faster, or at all in some rare cases like the Simulation-Based-Inference project. It simply was not feasible to do that research due to the large simulation data needed for inference training, and accelerating the simulations by orders of magnitude made it possible in matter of minutes--hours of computation rather than days--weeks.

%% %% The collaboration with INFN on the FEROCE project is another instance of a new challenge for me where I want to give my best and learn to work at a bigger scale with more colleagues, a bigger project and also a bigger work even on my own part, and I truly wish to take as much as I can from this experience too. These experiences have strengthened, and strengthening my ability to work with scientific software in multiple research fields, accelerate existing codebases, and approach research questions with both rigor and creativity.

%% I am particularly motivated by environments where computational approaches can make a real difference in advancing physical research—whether through large-scale simulations, efficient data analysis pipelines, or the integration of machine learning in inference tasks. I believe that working in an international research group would allow me to broaden my perspective, learn new methodologies, and contribute actively to collaborative projects.

%% Beyond the technical and academic aspects, I am also eager to immerse myself in a new cultural and scientific context. Having already experienced the value of collaboration and exchange during my studies in Padua, I am enthusiastic about the opportunity to take this further through an Erasmus+ traineeship, working alongside researchers from diverse backgrounds and experience the city.

%% With this traineeship, I hope to contribute my computational and theoretical skills to a stimulating research project, ideally as part of a team working on computational physics, complex systems, or scientific computing. I am flexible regarding the timing and duration of the traineeship.

%% Thank you very much for your consideration.


Dear Selection Committee,

My name is Alessio Tuscano and I am currently enrolled in the Master’s Degree in Physics of Data at the University of Padua. I am applying for the Erasmus+ Traineeship Programme to further develop my skills and contribute to research in an international, collaborative environment focused on computational physics, scientific high--performance computing, and complex systems.

During my academic journey, I have cultivated a strong interest in the intersection between theoretical physics and advanced computational methods. My studies and projects—often in collaboration with peers—have allowed me to translate physical models into efficient simulation code, optimize algorithms for parallel architectures, and apply modern data analysis techniques to scientific problems. Notably, my Bachelor thesis taught me how to build scientific datasets for deep learning, while recent individual and group projects have helped me refine my coding practices and manage larger, modular codebases.

I am passionate about making scientific code faster and more efficient, enabling researchers to achieve their goals more rapidly. For example, in a Simulation-Based-Inference project, accelerating simulations made previously infeasible research possible within practical timeframes.

I am motivated by environments where computational approaches advance physical research—whether through large-scale simulations, efficient data pipelines, or integrating machine learning in inference tasks. I believe that working in an international research group would broaden my perspective and allow me to contribute actively to collaborative projects.

Beyond the technical and academic aspects, I am eager to immerse myself in a new cultural and scientific context. Having experienced the value of collaboration during my studies in Padua, I look forward to extending this through an Erasmus+ traineeship.

With this traineeship, I hope to contribute my computational and theoretical skills to a stimulating research project, ideally as part of a team working on computational physics, complex systems, or scientific computing. I am flexible regarding timing and duration.

Thank you very much for your consideration.



\vspace{1em}
\noindent
Sincerely,\\[.4em]
Alessio Tuscano

\end{document}
